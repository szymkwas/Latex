\documentclass[11pt, a4paper]{article}

\usepackage[pl]{Szablon_raportu_AM_Latex} % [pl] or [en]

%%% Laboratory info %%%
\university{Politechnika Poznańska}
\faculty{Wydział Automatyki, Robotyki i Elektrotechniki}
\institute{Instytut Robotyki i Inteligencji Maszynowej}
\department{Zakład Sterowania i Elektroniki Przemysłowej}
\lab{Aplikacje mobilne i wbudowane dla Internetu Rzeczy}
\instructor{mgr inż. Adrian Wójcik}
\instructoremail{Adrian.Wojcik@put.poznan.pl}
\comment{Raport laboratoryjny}

%%% Report info %%%%

% Laboratory exercise:
\maintitle{Zadanie zaliczeniowe - serwer IoT }

% #1 author info
\firstauthor{Szymon Kwasiborski 140270, Miłosz Plutowski 140299}
\firstauthoremail{szymon.kwasiborski@put.poznan.pl, miłosz.plutowski@put.poznan.pl }

% #2 author info (leave blank if only one author)
\secondauthor{Jakub Gątarski 140241, Filip Kałużny 140 252}
\secondauthoremail{jakub.gatarski@put.poznan.pl, filip.kaluzny@put.poznan.pl}

% Report submittion 
\date{01-07-2021} 

\addbibresource{Szablon_raportu_AM_Latex.bib}

%%
%%

\begin{document}

%% First page %%
\mainpage
\newpage

%% Table of contents
\tableofcontents
\thispagestyle{fancy}
\newpage

\section*{Wstęp} \addcontentsline{toc}{section}{Wstęp}
Raport wykonanego projektu wykonano w oparciu o szablon pobrany z platformy uczelnianej eKursy \cite{ekursy}. Do jego wykonania posłużyliśmy się wiedzą nabytą podczas kursu: Aplikacje mobilne i wbudowane dla Internetu Rzeczy - laboratorium. \\
Poniższe sprawozdanie podzielono na cztery główne sekcje, podobne do tych używanych podczas wykonywania sprawozdań z poszczególnych instrukcji laboratoryjnych, w trakcie roku akademickiego.

\section{Opis specyfikacji}
Na samym wstępie opiszemy jakie wymogi (rzecz jasna poza \textbf{podstawowymi}) chcemy zrealizować w naszym projekcie. W późniejszych sekcjach raportu dotyczących opisu implementacji poszczególnych aplikacji, zaznaczymy czy zamierzona funkcjonalność (dany wymóg) została spełniona.
\begin{enumerate}
\item Stworzony system wykorzystywać będzie architekturę REST
\item Wszystkie trzy środowiska zachowają analogiczną architekturę oraz nazewnictwo metod
\item Kod źródłowy będzie zawierał komentarze według wspólnego standardu
\item Po uruchomieniu serwera odbędzie się automatyczne uruchomienie skryptów serwera
\item Każda z aplikacji umożliwiać będzie próbkowanie danych z okresem maksymalnie 100ms
\item Aplikacje serwera pozwalać będą na podgląd wszystkich wielkości fizycznych odczytanych z czujników
\item Podczas realizacji, implementacji wykorzystamy system kontroli wersji - GitHub
\item Aplikacja mobilna wykorzysta wzorzec architektoniczny zapewniający separację interfejsu użytkownika od logiki aplikacji
\item Aplikacja desktopowa wykorzysta wzorzec architektoniczny zapewniający separację interfejsu użytkownika od logiki aplikacji
\item Wszystkie stworzone aplikacji posiadać będą jednolitą szatę graficzną
\end{enumerate}


\newpage

\section{Implementacja systemu}

\subsection{Aplikacje serwera}

\subsection{Mobilna aplikacja klienta}

\subsection{Webowa aplikacja klienta}

\subsection{Desktopowa aplikacja klienta}
\newpage
\section{Wyniki testów i integracji systemu}

\newpage

\section{Wnioski i podsumowanie}


\newpage

\printbibliography[heading=bibintoc]

\end{document}